\bgroup
\def\arraystretch{1.5}
\begin{table*}[t]
\centering
{\small
	\begin{tabular}{|l|l|l|l|}
	\hline
	\multicolumn{2}{|l|}{\textbf{Approach}}                          
	& \textbf{Advantages}                                                                                            
	& \textbf{Disadvantages}                                                                                                                                                
	\\ \hline
	
	\multicolumn{2}{|l|}{Always Awake}                               
	& Highest Recall                                                                                                 
	& Highest power consumption                                                                                                                                             
	\\ \hline
	
	\multicolumn{2}{|l|}{Duty Cycling}                               
	& Significant power savings                                                                                      
	& \begin{tabular}[c]{@{}l@{}}Significant loss of event of interest recall\\ Power inefficient for infrequent events of interest\end{tabular}                            
	\\ \hline
	
	\multicolumn{2}{|l|}{Batching}                      
	& \begin{tabular}[c]{@{}l@{}}Significant power savings\\ Recall matching {\em Always Awake}\end{tabular}    
	& \begin{tabular}[c]{@{}l@{}}Significant delay between occurrence and detection of events\\ Power inefficient for infrequent events of interest\end{tabular} 
	\\ \hline
	
	\multicolumn{2}{|l|}{Computation Offloading}                     
	& Significant power savings                                                                                      
	& \begin{tabular}[c]{@{}l@{}}Programming complexity\\ Security concerns\\ Difficult to support multiple applications\end{tabular}                                       
	\\ \hline
	
	\multirow{2}{*}{\begin{tabular}[c]{@{}l@{}}Predefined Event\\ Wake-ups\end{tabular}} 
	& \begin{tabular}[c]{@{}l@{}}High-level\\ activity\end{tabular} 
	& \begin{tabular}[c]{@{}l@{}}Easy to use\\ High recall\\ Significant power savings\end{tabular}                  
	& Only usable by a small set of applications                                                                                                                      
	\\ \cline{2-4} 
	& \begin{tabular}[c]{@{}l@{}}Significant\\ Motion\end{tabular} 
	& \begin{tabular}[c]{@{}l@{}}Easy to use\\ High recall\end{tabular}       
	& \begin{tabular}[c]{@{}l@{}}High proportion of unnecessary wake-ups when event of interest\\ is infrequent\end{tabular}     
	\\ \hline
	\end{tabular}
}
\caption{Advantages and disadvantages of existing continuous mobile sensing approaches.}
\label{table:sensingApproachesSummary}
\end{table*}
\egroup

\section{\label{sec:Introduction}Introduction}

Today, smartphones and tablets are used primarily to run interactive
foreground applications, such as games and web browsers.  As a result,
current mobile devices are optimized for a use case where applications
are used intermittently during the day, in sessions that last for
several minutes.  For example, during her break Alice picks up
her phone, checks the weather, reads the latest news and plays a
game for a few minutes, and then puts it away to go back to work.  To
maximize battery life, current mobile devices are designed to go
into sleep state for most of the day when they are not supporting such
interactive usage.

Unfortunately, most mobile platforms are a poor match for a growing
class of mobile applications that perform continuous background
sensing.  Examples range from context-aware applications~\cite{baldauf2007survey,hong2009context}, such as
medical applications~\cite{hameed2003application,preuveneers2008mobile,tsai2007usability} that improve our
well-being to applications that use participatory sensing to get a
better understanding of the physical world, such as noise pollution
monitoring~\cite{maisonneuve2009citizen,maisonneuve2009noisetube} or 
traffic prediction~\cite{hull2006cartel}.  While the processing demands of
these applications are modest most of the time, they require periodic
collection of sensor readings.  An application that implements continuous 
sensing using an {\bf always awake} approach can achieve the highest possible 
accuracy because it receives all the sensor data.  However, the power
consumption of such an approach is very high.  

An alternative approach, {\bf duty cycling}, allows a device to sleep for fixed, usually
regular, periods of time.  An application can use the system's alarm
service to schedule regular wake-ups.  After a wake-up, the application
collects sensor readings for a short period of time, before
allowing the device to go back to sleep.  While duty cycling can achieve
significant power savings, it has 
several drawbacks.  While the device is sleeping,
events of interest will not be detected because the application does
not receive any sensor data.
Additionally, during the short time periods when the device is awake
and collecting sensor readings, the events of interest might not
occur.  These power transitions are wasteful and expensive.  Another 
drawback of this approach is that it does not scale well with
multiple applications.  It is possible for mutually unaware
applications to implement conflicting wake-up schedules, resulting in
little or no sleep.

{\bf Sensor data batching} is an extension of duty cycling in which
the device hardware is able to collect sensor readings while the main
processor is asleep.  Upon wake-up, the entire batch of sensor data is
delivered to the application(s) that requested data from that
specific sensor.  Unlike duty cycling, batching requires hardware
support, but it enables applications to receive all the sensor data, 
and hence provides detection accuracy similar to the always awake 
approach.  However, batching affects detection 
timeliness.  Applications may detect events of interest many seconds 
after they actually occurred.  Also, as with the duty cycling
approach, the device may wake up to find out that no events of
interest have occurred in the current batch, which is wasteful.

To improve support for continuous sensing applications the research
community has proposed the use of {\bf fully progra\-mmable heterogeneous
architectures}~\cite{reflex,littlerock,turducken}.  In these
approaches, developers offload their own sensor data analysis 
algorithms to a low-power processor (or a hierarchy of processors), 
enabling arbitrarily rich sensing applications.  When the code running 
on the low-power processor detects the occurrence of an event of 
interest, it proceeds to wake up the phone and passes control to the 
rest of the application.  While the ability to run custom code on the 
low-power hardware provides great flexibility, the significant complexity
inherent in this approach has so far prevented its adoption in
commercial devices.  Furthermore, this approach may create security concerns
because applications are allowed to run arbitrary code on the
peripheral processor.  Additionally, since each application is written independently,
this approach makes it hard for multiple applications to program or
use the same sensor.  

Instead, smartphone manufacturers, realizing the
potential of sensing applications, have recently incorporated
low-power processors into their architectures, but have limited
application developers to APIs that provide a limited set of 
{\bf predefined event detection} algorithms that can be used as wake-up
conditions~\cite{androidMotionSensors,coreMotion}.  For example, 
the Motorola Moto X~\cite{motox,x8mobile} uses a dedicated natural language processor
to wake up the device when the user says the phrase ``OK Google Now''.  Also, 
a contextual processor is used to turn on a part of the screen
when the mobile device is taken out of a pocket, or start the camera
application when it detects two twists of the wrist.  Additionally, 
Android 4.4~\cite{android4.4} and Apple's CoreMotion~\cite{coreMotion} 
framework supports specific wake-up
events based on detecting ``significant motion'' or steps, if the device
has the appropriate sensors.  However, application developers have no
control over the underlying activity detection algorithms, including
customization of parameters.  This approach is easy to use, and it
can provide good energy savings.  However, it only works for applications that
can take advantage of the predefined set of activities, limiting the
number and types of applications that can be supported.

This paper proposes {\em Simple Configurable Classifiers} (SCC), a new
approach for continuous mobile sensing that splits the work of
energy-efficient event detection between the platform and the
application developer.  In the new approach, the platform implements
common sensor data processing algorithms (e.g., windowing, noise 
reduction, feature extraction, admission control) that execute on a 
low-power sensor hub, and application developers construct simple 
classifiers for events of interest by selecting among the set of
pre-defined common processing algorithms and tuning their
parameters.  The simple classifiers execute on the
low-power sensor hub and, when events of interest are detected, the
main processor is woken up and the rest of the application code is
invoked.
  
We argue that SCC provides a better
balance between flexibility and ease of deployment than existing
approaches.  The approach empowers developers to create a broad set of
event classifiers that can be used to detect a wide range of
activities.  Simultaneously, these classifiers are significantly
easier to program compared to fully programmable offloading.
Moreover, since the algorithms are pre-specified, their
implementations can be optimized (by the device manufacturer) for each
low-power processor, improving application portability between
devices.  

To evaluate the benefits of SCC, we developed applications that use 
accelerometer readings or audio data to detect several events of 
interest.  Additionally, we built a prototype implementation that extends 
a Nexus 4 phone with a low-power sensor board.  To enable us to 
conduct controlled and repeatable experiments,
we mounted the prototype on an AIBO ERA 210 robot dog.  Simulations 
conducted on accelerometer traces collected with this setup,
as well as accelerometer traces collected from human subjects and audio
traces collected in various environments, show that
SCC can reduce the average energy required to run continuous
sensing applications by up to 94\%, while matching the detection recall
and precision of an approach that keeps the phone awake at all times.
Moreover, for most of our usage scenarios, the SCC approach
achieves over 90\% of the power savings achieved by a ``perfect''
wake-up mechanism, indicating that an implementation that supports
custom code offloading will achieve only marginal additional
improvements.

The rest of this paper is organized as follows.  Sections~\ref{sec:conjecture}
and~\ref{sec:prototype} describe our new approach for continuous sensing
and our prototype.  Sections~\ref{sec:experimentalSetup} 
and~\ref{sec:results} present our
experimental setup and the results from our evaluation.  Finally,
Sections~\ref{sec:related} and~\ref{sec:conclusion} describe our work
in the context of related work and conclude the paper.

