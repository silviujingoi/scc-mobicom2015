\section{Related Work}\label{sec:related}

This work is an extension of our position paper~\cite{jingoiSCCposition}, 
where we provided preliminary results showing the potential benefits
of simple configurable classifiers. Our current work supplements our 
previous findings through the development of
a prototype system and an extensive evaluation using accelerometer 
and audio applications.

The idea of waking up a device when an event of interest occurs has been 
around since the inception of mobile phones. The phone's radio 
transceiver wakes up the device when an incoming call or a text message 
is received~\cite{gobi}. Wake on Wireless~\cite{shih2002wake} extended this idea by 
augmenting a PDA with a low-power radio that would send a wake-up message 
when an incoming call is received. Similarly, Wake on WLAN
~\cite{mishra2006wake} allows remote wake-up of wireless networking
equipment.

Turducken~\cite{turducken} generalizes the ``wake on event of interest'' 
approach to several types of applications and to multiple components 
operating at increasingly small power-levels. Little Rock~\cite{littlerock} 
applies Turducken's multi-tiered architecture to sensing on mobile 
devices. Reflex ~\cite{reflex} complements the idea proposed by Turducken 
by providing a shared memory abstraction to be used by the different 
processors.

Simple Configurable Classifiers differ from these approaches by hiding the heterogeneous 
nature of the system from the application developer. Creating an 
application that makes use of SCC does not require 
the developer to create any code that will run on the low-power 
processor(s).  We limit the available interface in order to increase 
portability, 
while still achieving the majority of the potential power savings.

Smartphone manufacturers have started to incorporate low-power
processors into their architectures, but have only implemented limited
APIs that provide fixed functionality.  Apple's M7 and M8 motion
co-processors are used to collect, process, and store sensor data 
even while the main CPU is asleep and applications can retrieve 
historical motion data via the CoreMotion API~\cite{coreMotion}.  Some
recent Android devices allow
batching of sensor readings~\cite{android4.4}, and the
Motorola Moto X provides recognition for a small number of predefined
activities that can be used as wake-up 
conditions~\cite{motox,x8mobile}.  While
these wake-up conditions work well for some applications, they are
inefficient for many other types of applications that are not
interested in the set of predefined activities.

Most of the previously noted works focused on system architecture 
modification in order to lower the cost of sensing. Alternative 
approaches have also been explored. Ace~\cite{ace} is a middleware that 
supports continuous context-aware applications while mitigating sensing 
cost for acquisition of context attributes (such as AtHome and IsDriving). 
It achieves power savings when multiple applications request strongly 
correlated context attributes. Additionally, it can reduce power 
consumption when a ``cheaper'' sensor exists, which can determine the 
value of a different context attribute that has a strong correlation with 
the requested context attribute (e.g. use the accelerometer to check if 
the user is jogging instead of using the GPS to determine if the user is 
at work). A middleware such as Ace is a great example of a library that 
can run on top of a Smartsensor architecture and achieve additional power savings. Sensor fusion 
has also been an active focus of related research. Data from multiple 
sensors can be used to increase context-awareness in mobile devices 
\cite{gellersen2002multi,biegel2004framework}.

While our focus was on power-efficient acquisition of sensor data, next 
generation mobile perception applications face related problems regarding 
partitioning of application code. MAUI~\cite{maui} enables fine-grained
energy-aware offload of mobile application code to remote 
servers. Similarly, Odessa~\cite{ra2011odessa} uses code-offloading to 
address the issue of processing excessive amounts of sensor data on 
resource constrained mobile devices.