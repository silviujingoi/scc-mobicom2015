\bgroup
\def\arraystretch{1.5}
\begin{table*}[t]
\centering
{\small
	\begin{tabular}{|l|l|l|l|}
	\hline
	\multicolumn{2}{|l|}{\textbf{Approach}}                          
	& \textbf{Advantages}                                                                                            
	& \textbf{Disadvantages}                                                                                                                                                
	\\ \hline
	
	\multicolumn{2}{|l|}{Always Awake}                               
	& Highest Recall                                                                                                 
	& Highest power consumption                                                                                                                                             
	\\ \hline
	
	\multicolumn{2}{|l|}{Duty Cycling}                               
	& Significant power savings                                                                                      
	& \begin{tabular}[c]{@{}l@{}}Significant loss of event of interest recall\\ Power inefficient for infrequent events of interest\end{tabular}                            
	\\ \hline
	
	\multicolumn{2}{|l|}{Batching}                      
	& \begin{tabular}[c]{@{}l@{}}Significant power savings\\ Recall matching {\em Always Awake}\end{tabular}    
	& \begin{tabular}[c]{@{}l@{}}Significant delay between occurrence and detection of events\\ Power inefficient for infrequent events of interest\end{tabular} 
	\\ \hline
	
	\multicolumn{2}{|l|}{Computation Offloading}                     
	& Significant power savings                                                                                      
	& \begin{tabular}[c]{@{}l@{}}Programming complexity\\ Security concerns\\ Difficult to support multiple applications\end{tabular}                                       
	\\ \hline
	
	\multirow{2}{*}{\begin{tabular}[c]{@{}l@{}}Predefined Event\\ Wake-ups\end{tabular}} 
	& \begin{tabular}[c]{@{}l@{}}High-level\\ activity\end{tabular} 
	& \begin{tabular}[c]{@{}l@{}}Easy to use\\ High recall\\ Significant power savings\end{tabular}                  
	& Only usable by a small set of applications                                                                                                                      
	\\ \cline{2-4} 
	& \begin{tabular}[c]{@{}l@{}}Significant\\ Motion\end{tabular} 
	& \begin{tabular}[c]{@{}l@{}}Easy to use\\ High recall\end{tabular}       
	& \begin{tabular}[c]{@{}l@{}}High proportion of unnecessary wake-ups when event of interest\\ is infrequent\end{tabular}     
	\\ \hline
	\end{tabular}
}
\caption{Advantages and disadvantages of existing continuous mobile sensing approaches.}
\label{table:sensingApproachesSummary}
\end{table*}
\egroup

\section{Background}
\label{sec:background}

This section overviews existing approaches for continuous mobile
sensing.  The availability of some of these approaches depends on the
mobile device's hardware and operating system.
Table~\ref{table:sensingApproachesSummary} summarizes the main
properties of the various approaches.

\subsection{Always Awake}

The basic approach for continuous sensing is to keep the mobile device
\emph{always on}.  As an example, on the Android operating system, the
sensing application acquires a wake-lock to prevent the device from
sleeping and registers a listener for a specific sensor.  The benefit
of this approach is that it achieves the highest possible accuracy
because all the sensor data is delivered to the application.  However,
the device cannot be placed in a sleeping state and hence, the power
consumption of this approach is high.

\subsection{Duty Cycling}

To avoid keeping the device awake for extended periods of time,
\emph{duty cycling} allows a device to sleep for fixed, usually
regular, periods of time.  An Android application can implement
duty cycling by scheduling a wake-up alarm using the system's alarm
service and releasing all its current wake-locks.  When no
application has a wake-lock, the operating system can power down the
processor.  When the wake-up alarm fires, the processor wakes up and
notifies the application.  Next, the application collects sensor
readings for a short period of time, and then reschedules another
alarm.

Duty cycling has several drawbacks.  While the device is sleeping,
events of interest will not be detected because the application does
not receive any sensor data that were produced at those times.
Additionally, during the short time periods when the device is awake
and collecting sensor readings, the events of interest might not
occur.  These power transitions are wasteful and expensive.
Unfortunately, finding a good balance between accuracy and power
savings depends on the specific application and the user's behavior.
Another drawback of this approach is that it does not scale well with
multiple applications.  It is possible for mutually unaware
applications to implement conflicting wake-up schedules, resulting in
little or no sleep.

\subsection{Sensor Data Batching}

An extension to duty cycling is \emph{sensor data batching} in which
the device hardware is able to collect sensor readings while the main
processor is asleep.  Upon wake-up, the entire batch of sensor data is
delivered to the application(s) that registered a listener for that
specific sensor.  Unlike duty cycling, batching requires hardware
support (e.g., it is currently only available on the 
Google Nexus 5 and 6, and the iPhone 5S, 6 and 6S), 
but it enables applications to
receive all the sensor data, and hence provides detection accuracy
similar to the Always Awake approach.  However, batching affects
detection timeliness.  Applications may detect events of interest many
seconds after they actually occurred.  Also, as with the duty cycling
approach, the device may wake up to find out that no events of
interest have occurred in the current batch, which is wasteful.

\subsection{Computation Offloading}
\label{subsec:computationOffloading}

Generalizing batching, a low-power peripheral processor can be used to
collect and \emph{compute} on sensor data while the main processor is
asleep~\cite{reflex,turducken}.  This approach allows developers to
offload their own sensor data analysis algorithms to the peripheral
processor, enabling arbitrarily rich sensing applications.  However,
developers are exposed to the full complexity of the architecture,
making application development more difficult.  Even simple
sensor-driven applications have to be refactored into distributed
programs, and the code that is offloaded has to be chosen
carefully so that it can run in real time.  The latter is complicated
because it depends on the type and functionality of the peripheral
processor that is available.  As a result, multiple versions of the
application may be required to handle hardware variations across
phones.  Furthermore, this approach may create security concerns
because applications are allowed to run arbitrary code on the
peripheral processor.  Additionally, since each application is written independently,
this approach makes it hard for multiple applications to program or
use the same sensor.

\subsection{Activity Detection}

Given the challenges with full computation offloading, recent mobile
devices limit offloading to a small, predefined set of \emph{activity
detection} algorithms.  For example, the Motorola Moto 
X~\cite{motox,x8mobile}
uses two low-power peripheral processors to wake up the device when
certain events occur.  A dedicated natural language processor is used
to wake up the device when a user says the phrase ``OK Google Now''.  Also, 
a contextual processor is used to turn on a part of the screen
when the mobile device is taken out of a pocket, or start the camera
application when it detects two twists of the wrist.  Additionally, 
Android 4.4~\cite{android4.4} and Apple's CoreMotion~\cite{coreMotion} 
framework supports specific wake-up
events based on detecting ``significant motion'' or steps, if the device
has the appropriate sensors.  However, the application developer has no
control over the underlying activity detection algorithms, including
customization of parameters.  This approach is easy to use, and it
provides good energy savings because the device only wakes up when the
event of interest occurs.  However, it only works for applications that
can take advantage of the predefined set of activities, limiting the
number and types of applications that can be supported.

