\section{Simple Configurable Classifiers}
\label{sec:conjecture}

Continuous mobile sensing approaches have to address 
two main constraints: maximizing detection accuracy
and minimizing energy consumption.  Users expect very
high precision and recall (95\% or more) and user
experience is adversely affected when the application
misses or over-reports events of interest.  Achieving 
both high recall and precision requires 
highly specialized algorithms tuned to the event of 
interest.  Specifically, getting the last few percentages
points in precision and recall is difficult and requires
complex algorithms and fine parameter 
tuning.  This effectively means that an algorithm only
works for a specific event.

However, the two main goals (energy efficiency and high
precision and recall) do not have to be achieved by a 
monolithic algorithm.  A preliminary stage can be used
to achieve the majority of available energy savings and 
the main stage can target high detection precision and recall.

We argue that continuous mobile sensing applications should be
structured as a pair of classifiers of increasing complexity: a simple
high recall and moderate precision classifier that runs continuously
on low-power hardware, and acts as an energy efficient wake-up
mechanism for a higher complexity classifier that runs on the main CPU
and provides both high recall and high precision.

We conjecture: 1) that it is possible to implement simple classifiers for a 
wide range of applications by configuring a small set of common processing 
algorithms; and  2) that this approach will achieve comparable energy 
savings to an alternative implementation that supports full programmability.

\subsection {Approach}

\begin{figure}[t]
	\centering
	\includegraphics[width=3.1in]{architecture.png}
	\caption{Proposed system architecture.}
    \label{fig:architecture}
\end{figure}


\begin{figure*}[t]
{\small
	\begin{verbatim}
	
# segment sensor data in windows of size 512 and apply a low-pass filter
$LPF = Filter(type=lowpass, cutoff=1000, rolloff=6, size=512)

# feature extraction: zero-crossing rate over windows of size 512
$ZCR = Aggregate(type=zcr, size=512)

# threshold-based admission control to be applied to the extracted zero-crossing rate
$AC = AdmissionControl(type=threshold, min=100, max=200)

# Define data stream flows
$DATA > $LPF > $ZCR > $AC

Assert($AC)
	\end{verbatim}
}
	\caption{Graph definition example}
    \label{fig:sccLanguageSample}
\end{figure*}


\iffalse
\begin{figure*}[t]
{\small
	\begin{verbatim}
# segment sensor data and apply a low-pass filter
$DATA > window type=rect size=512 overlap=256 | filter type=lowpass cutoff=1000 rolloff=6 > $lpfwin

# extract the zero-crossing rate and perform threshold-based admission control
$lpfwin > extract type=zcr | admissioncontrol type=threshold min=100 max=200 > $aczcr

# check if admission control passed
assert $aczcr

# pass filtered sensor data to application
return $lpfwin
	\end{verbatim}
}
	\caption{Graph definition example}
    \label{fig:sccLanguageSample}
\end{figure*}
\fi

{\em Simple Configurable Classifiers} (SCC) is a new approach for
continuous mobile sensing that divides the responsibility of
energy-efficient event detection between the platform and
the application developer.  The platform implements common sensor data
processing algorithms that execute on a low-power sensor hub.
Applications construct simple classifiers for events of interest by
defining a directed acyclic graph using a set of pre-defined 
algorithmic building blocks.  The vertices
of the graph represent the data processing algorithms, and the 
edges of the graph represent data flows; i.e. the output of an algorithm 
becomes the input of one or more other algorithms.  This classifier 
executes on the low-power sensor hub and, when events of interest 
are detected, the main processor is woken up and the application 
code is invoked.

Figure~\ref{fig:architecture} shows the architecture of a
system that features SCC.  Applications interact with a sensor 
manager to define simple configurable classifiers.  We anticipate 
that the platform will need to implement algorithms
that perform windowing (e.g. rectangle, Hamming), filtering (e.g.
noise reduction, low-pass, high-pass), transforms (e.g. Fast Fourier
Transform), feature extraction (e.g. zero crossing rate) and admission
control (e.g. thresholds).  The figure also shows that the 
architecture supports recognition libraries that–]buzzpunk FaceIt Player 10 points 3 hours ago 
V
encapsulate the functionality of SCC to provide simple
wake-up conditions and event detection for a large number of 
activities. 

  

\subsection{Language}

To define a SCC, developers can use a language similar to shell
scripting.  Figure~\ref{fig:sccLanguageSample} shows a sample
definition of a simple configurable classifier.  The first section is used
to define and configure the necessary algorithms.  In the second section the
{\em greater-than} character is used to define data flows between the 
various algorithms. 

\subsection{Advantages}

SCC has many benefits:

 {\bf Lower programming complexity}  Programming complexity is
  decreased because application developers can use the pre-defined
  processing algorithms, rather than implementing them themselves. 
  Application developers need to select appropriate processing 
  algorithms for their pipelines and configure their parameters. This 
  is not unlike how shell programming is used to build complex 
  pipelines by reusing existing command-line applications.
  

{\bf Better predictability} It 
  creates predictable performance for the algorithms executed on the
  low-power processor. Profiling each algorithm will provide metrics
  about computational and time requirements, as well as power
  consumption.  Predictable performance is important in order to
  support multiple concurrent classifier executions or applications
  that make use of multiple sensors.


{\bf Better security} Providing access to these algorithms via
  an API has significant security advantages over the fully
  programmable offloading approach because application developers
  cannot execute arbitrary code on the peripheral processor. 


{\bf Improved Portability} Programmers do not have to be
  aware of the specifics of the underlying hardware, nor create a version
  for every type of sensor hub. Since the algorithms are
  pre-specified, device manufacturers can optimize their
  implementations for each low-power processor and sensor
  implementation.



\subsection{Current Challenges}
\label{sec:currentChallenges}

{\bf Identifying processing algorithms} Defining the appropriate
set of algorithms that should be included in the API and executed on
the low-power processor for each sensor is a key challenge. First,
there is a trade-off between algorithm generality and accuracy.
Simple generic algorithms can support a large set of
applications, albeit no specific application is likely to experience
optimal performance.  Conversely, a highly specialized algorithm may
provide optimal performance but is only applicable to a limited set of
applications.  Second, there is also a trade-off between algorithm
complexity and power savings.  More complex algorithms can reduce
energy consumption by preventing unnecessary wake-ups due to increased
accuracy.  On the other hand, more complex algorithms have higher
computational demands, which require a larger and hungrier peripheral
processor.  

While determining the complete set of 
algorithms to be include as part of the platform is beyond the
scope of this paper, we anticipate that it will include algorithms 
such as the ones listed in Section~\ref{sec:sensorDataAlgorithms}.
The list contains algorithms that were needed in order to construct
the simple classifiers were examined in the evaluation.  To 
determine an extensive list of 
algorithms that should be included for each sensor, a wide variety
of sensor-driven applications needs to be analysed.


{\bf Access to sensor data} A related challenge is determining what
data the sensor hub should pass to the application following a
wake-up.  Some applications may be interested in the raw sensor data,
while others may want to use the filtered data or extracted features.
Our prototype includes an across-the-board solution that allows the 
applications to specify what data they are interested in via the API.


{\bf Configuring classifiers} After designing the simple 
configurable classifiers we faced another challenge in determining 
the optimal parameter configurations for each algorithm.  Since our
classifiers were based on previously developed applications, we were
able to reuse parameters for algorithms that were common for both the
application and the classifier.  For other parameters, we ended up
performing searches of the parameter space and choosing the parameters
that generated the highest recall and precision.

We believe application developers may
face challenges in selecting the optimal algorithms and configuration
parameters for their simple classifiers.  Given feedback from the
application, self-learning mechanisms may be able to determine optimal
configuration parameters for the algorithms used.  It is easy to 
imagine an application notifying the
sensor hub about wake-ups when events of interest were not actually
detected (i.e. false positives).  However, it will be more difficult
to automatically identify events of interest missed by the classifier
running on the low-power node (i.e. false negatives).

{\bf Sizing} When creating the sensor node of the prototype 
implementation we evaluated two micro-controllers having 
different power consumption levels.  We noticed that the lower 
power micro-controller was not able to run some algorithms (such
as Fast Fourier Transforms) in real-time. Determining the optimal 
number, type and size of
processors to include in the sensor hub is an open research question.
Each sensor (or small group of related sensors) may be supported by
its own dedicated low-power processor.  Alternatively, a larger
processor could be used to serve the entire sensor hub.  Identifying a
sweet spot between the maximum number of concurrent algorithm
executions, energy budget, cost and physical size of the sensor node
is an open challenge.


\subsection{Further Challenges}
\label{sec:furtherChallenges}

{\bf Multi-application support} Another challenge lies in supporting
multiple concurrent applications while still maintaining predictable
performance.  This challenge is not unique to our approach; however,
given that the algorithms in our approach are pre-defined, they can be
profiled in terms of computational power requirements.  A resource
manager would be required to orchestrate and synchronize concurrent
requests from multiple applications.

{\bf Sensor fusion} Fusing inputs from multiple sensors is a common
technique used for improving the accuracy of sensing applications.
Whether low power sensor hubs should include support for sensor
fusion, however, is not clear.  On the one hand, such support could
increase energy efficiency by reducing the occurrence of unnecessary
wake-ups.  On the other, sensor fusion tends to be application specific
and the added complexity may negate any energy benefits.
