\section{Experimental Setup}
\label{sec:experimentalSetup}

We used trace based simulation to evaluate our accelerometer and 
audio applications, running under several sensing configurations.

\subsection{Trace Collection}
\label{sec:traces}

{\bf Audio traces} 
We collected three half-hour audio traces in
different environments: an office, a coffee shop and outdoors.  We
used audio mixing software to add audio events of interest to the
collected traces.  The audio events of interest include music (5\%
of each trace), speech (5\% of each trace), and sirens (2\% of each 
trace).  The events of interest were randomly selected from a library
of audio files.

{\bf Human accelerometer traces}
In addition, we collected six hours of accelerometer traces from
three different individuals while they perform routine daily
activities: morning commute using public transit, working in a retail
store, and working in an office.  Between 20\% and 37\% of each trace
is spent walking.

{\bf Robotic accelerometer traces} We collected synthetic traces by having 
a robot perform multiple runs with a prototype smartphone attached 
to its back.  For each run, the robot logged the start and end of each 
action, which we use as the ground truth for our experiments.  The 
smartphone ran an application that kept the device always awake and 
continuously recorded accelerometer readings for all three axes. 

In each run, the robot performed five different actions: standing
idle, walking, sit-to-stand transitions, stand-to-sit transitions, and
headbutts.  We created runs with three different levels of activity.
Runs in groups 1, 2 and 3 spent 90\%, 50\% and 10\% of the time
standing idle, respectively. The reminder of the time was allocated as
follows: 73\% for walking, 24\% for transitions between sitting and
standing, and 3\% for headbutts.  This setup allows us to experiment
with detecting actions that are common, somewhat frequent, and rare.
In total, the robot executed 18 different runs: 9 for group 1, 6
for group 2 and 3 for group 3.  We generated more runs for groups 1
and 2 because of the lower activity levels compared to group 3. To
eliminate bias, the list of actions was generated randomly for each
run, based on the expected probabilities of each action occurring.


\subsection{Test Platform}
\label{sec:testPlatform}

To enable us to conduct controlled and repeatable experiments, we
mounted the prototype smartphone on the back of an AIBO ERA 210 robot
dog (see Figure~\ref{fig:aibo}).  Because the robot's actions can be
scripted, this setup provides an efficient and reliable way to
determine ground truth.  In contrast, labelling data collected from
human subjects with ground truth is error prone and labour intensive.

While our robotic testbed allows us to run 
live experiments, we chose instead to use trace-based simulation for 
several reasons.  First, it
took the robot close to an hour to complete a single experiment.  A
thorough exploration of the configuration space of the various sensing
approaches we consider would have required months of continuous
live experiments.  Moreover, taking fine grain power consumption
measurements while the robot is in motion is not trivial.


\subsection{Configurations}
\label{sec:configurations}
We used the simulator to evaluate the recall and precision of 
our applications under the following configurations:

\begin{itemize}

\item {\bf Duty Cycling} The applications wake-up at fixed time 
intervals to collect sensor data for 4 seconds and run the event 
detection algorithms.  If an action is detected, the phone is 
kept awake for another 4 seconds, and goes to sleep otherwise. 

\iffalse
\textbf{Duty Cycling} We modified the applications so that they check
sensor readings periodically and then put the phone to sleep.  On
wake-up, the phone is kept awake for 4 seconds in order to collect
sensor data.  If an action is detected, the phone is kept awake for
another 4 second, and goes to sleep otherwise.  This software only
implementation runs on any mobile device and does not require special
hardware support.
\fi

\item {\bf Batching} The applications wake-up at fixed time intervals to receive the batch of sensor data and run the event detection algorithms.

\iffalse
\textbf{Batching} This configuration emulates hardware support
available on the Nexus 5 for collecting and batching accelerometer
readings while the main processor sleeps.  The application wakes up
periodically, reads the batch of sensor readings, runs the detection
algorithm, and goes back to sleep.
\fi

\item {\bf Generic Wake-up Condition} This configuration simulates the Android's built-in significant motion detector.
We constructed simple classifiers to wake up the device and run the applications when significant activity is detected (significant acceleration or sound).

\item {\bf Simple Configurable Classifier} For each of the 
applications, we constructed simple event classifiers to 
wake up the device and run the event detection 
applications.  A more detailed description of the 
classifiers is available in Section~\ref{sec:classifiers}.

\item {\bf Oracle} A hypothetical ideal implementation that only wakes up
when the event of interest occurs.  Such a wake-up condition would
achieve perfect detection precision and recall, with
the lowest possible power consumption. The difference between the
power consumption of this method and the SCC configuration
provides an upper bound on the potential additional benefits of custom
code offloading.
  
\end{itemize}

\subsection{Metrics}

For each sensing approach and trace, we measured the amount of sleep
and awake time, the total number of wake-up events, and the recall and
precision of the application.  Finally, we used an energy model
derived from measurements of our prototype to estimate
the average power consumption.  For the {\em Duty Cycling} experiments, 
the power model accounts only for the energy
consumption of the Nexus 4.  For {\em Batching} and {\em Generic Wake-up Condition},
the model also includes the cost of the low-power TI MSP430
micro-controller.  Finally, experiments configured to use {\em SCC} include the cost of 
either the TI MSP430 or the TI Stellaris
LM4F120H5QR, depending on the complexity of the algorithms used by the SCC.



